\chapter{Primera y segunda forma fundamental}

\section{¿?}
La primera forma fundamental da la geometría de la superficie. Podemos por
ejemplo utilizarla para calcular la longitud de curvas en $S$. Sea
$\alpha\colon I\subseteq\R\to S\subseteq\R^3$ una curva en $S$. Entonces
\[
	L(\alpha)=\int_I\sqrt{I_p(\alpha'(t))}dt.
\]

\index{Dominio}
También podemos utilizar la primera forma fundamental para el cálculo de áreas.
Recordemos que un \textbf{dominio} en una superficie $S$ es un abierto conexo
$D\subseteq S$ tal que $\partial D$ es la imagen del círculo $S^1$ por un
homeomorfismo diferenciable a trozos. Una \textbf{región} $R$ es la unión de un
dominio $D$ junto con su frontera $\partial D$. Definimos entonces
el área de un dominio $R$ como
\[
	\text{área}(R)=\int_Q\|X_u\times X_v\|dudv,
\]
donde $Q=X^{-1}(R)$. Esta definición no depende de la parametrización $X$ pues
si $Y\colon V\to Y(V)$ es otra parametrización tal que $R\subseteq X(U)\cap Y(V)$, 
la fórmula de cambio de variables implica que
\[
	\int_{X^{-1}(U)}\|X_u\times X_v\|dudv=\int_{Y^{-1}(R)}\|Y_u\times Y_v\|dudv.
\]

Observemos que como
\[
	\|X_u\times X_v\|=\sqrt{\langle X_u,X_u\rangle\langle X_v,X_v\rangle-\langle X_u,X_v\rangle^2}=\sqrt{EG-F^2},
\]
se tiene la siguiente fórmula:
\[
	\text{área}(R)=\int_{X^{-1}(U)}\sqrt{EG-F^2}dudv.
\]




\section{Superficies mínimas}



